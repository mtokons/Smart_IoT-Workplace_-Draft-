\chapter{Conclusions}
\textit{This chapter presents the conclusions of research by including a summary of the most important findings of the thesis as a whole. The conclusions that have been drawn are linked to Schneckenhaus system and are also in line with our research goal of having a clear understanding of what causes stress in the workplace, how Stress Overflow affects mental health, how to manage stress in the workplace.}
\vspace{5mm}

Based on the combination of our experimental study done a persona who has stress disability with the stress management system and literature review, I have reached the following conclusions. My experimental data is reliable to the extent that it helped stress overflowed people to attaining thesis aim which was to management of stress at the workplace. My research by considering  Stress Overflowed personnel made it even clearer to when and how they need to manage stress. This gave clear guidance towards answering the research question. The theoretical framework assisted me to understand what causes stress and how it can be managed. Whereas the practical findings allowed me to see this from stress disability people perspectives and the analysis created a link between the theories and the practical system development findings and summarized what the management mechanisms of stress at the workplace will help to catalogue stress anxiety, therefore it helps special needs people to become more confident in knowing their limits and taking measure at the right time which leads to improving the quality of work and health and empower them to perform more.

The aim of this research is to complement existing occupational stress support people find in the workplace. We have developed a stress management system "Schneckenhaus" which provides a buddy persona across multiple platforms including IoT devices, chatbots, Dashboard. Using conversation models, Sense of Coherence and support, we have implemented an initial phase of our system which has shown promising results.

Using a Schneckenhaus system for mental health counselling can provide many benefits for the user. The system can give a user instant information. By incorporating mental health screening tools into a Schneckenhaus \acf{UI} app, the user can have a more interactive and user-friendly experience. Research has shown users find system “safe” and easy to talk to. Schneckenhaus system can create another option for users who do not want to receive face to face treatment, however, there are many ethical aspects to consider.
 
Now I discuss with persona character Jan's and get his opinion regarding the final remarks regarding my system and application which we develop throughout our research. He said that the application is fun for him to manage stress at the workplace. The data in the Stress Management system is that using the Smart\acs{IoT} devices will help him to find a better rhythm and taking breaks before he experiences with full Stress Overload, therefore he can reduce the overall stress level throughout the days. Based on his last job, 07:00 am was the time of his work at the office started and at 14:00 pm he leaves office for home. For future workplaces he might start with shorter workdays based on his activity dashboard data, also 4 to 5 hours a day, to better stabilize his chronic condition.
 
To sum up: while many employees struggle with stress,  in the worst cases it leads to uncertainties and severe impairments on their health and performance.  The main situations that generate stress are likely uncontrollable,  unpredictable,  and are not known. With today’s stressful environment with increasing numbers of stress-related diseases, it would be very interesting and helpful to implement this Schneckenhaus system. This would allow people to keep track of their stress, to be able to gain more knowledge and learn how to control it. Which hopefully could bring a more healthy life to many stressful individuals.

\epigraph{Remember that stress doesn’t come from what’s going on in your life. It comes from your thoughts about what’s going on in your life.}{\textit{Andrew J. Bernstein \\ \citep{Nelson2016ProphetsSeminar}}}



