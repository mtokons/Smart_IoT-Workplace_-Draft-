\chapter{Conclusions}
\textbf{This section should have a summary of the whole project.  The original aims and objective and whether these have been met should be discussed. It should include a section with a critique and a list of limitations of your proposed solutions.  Future work should be described, and this should not be marginal or silly (e.g.\ add machine learning models).  It is always good to end on a positive note (i.e.\ `Final Remarks').}


 
6. Concluding Remarks 
This  chapter  presents  the  conclusions  of  our  research  by  encompassing  a  summary  of  the 
most important results from the whole thesis. The conclusions made are linked to our analysis 
and are also in line with our research aim which is to have a clear understanding of what 
causes stress at a multinational company such as Volvo trucks AB Umeå and how Stress by 
the employees as well as the company’s management are managed or handled.    
 
Based on the combination of our empirical study done at Volvo Trucks AB Umeå and our 
literature review we have reached the following conclusions. Our empirical data is reliable to 
the extent that it helped us in attaining our aim which was to find the causes and management 
of stress by both the employees and managers at Volvo Trucks AB Umeå. Our research by 
considering  Volvo Trucks  AB Umeå personnel  made it even  clearer to  what  the causes of 
stress  are  in  an  actual  company  and  how  the  employee  as  well  as  management  deals  with 
them. This gave us a clear guidance towards answering our research question. The theoretical 
framework assisted us to understand what causes stress and how it can be managed. Where as 
the  empirical  findings  allowed  us  to  see  this  from  Volvo  Trucks  AB  Umeå  employees’ 
perspectives and the analysis created a link between the theories and the empirical findings 
and summarized what the causes and management mechanisms of stress at a manufacturing 
multinational company such as Volvo Trucks AB Umeå are.   
 
6.1 Conclusions            
Our research using qualitative methods to understand the causes of stress at the work place 
particularly at Volvo Trucks AB Umeå and also observe the management mechanisms applied 
by both the employees and the management at Volvo Trucks AB, Umeå can be concluded by 
our findings and analysis of our results are summarized as follows. 
After conducting this research we can conclude that regardless of the employees’ job level, 
position or belongings of department, employees at Volvo trucks AB Umeå do feel stressed 
like other companies’ employees do. Employees feel stress because of various factors, mainly 
on job and off job reasons. But not all the reasons are directly linked to the issues that create 
stress at the workplace. We think that the stress level among employees at Volvo trucks AB 
Umeå is normal and it is not more than the stress that people feel outside the organization. 
With regards to theories and information collected from the respondents during interviews, 
we come up with three main conclusions.  
 
Conclusion 1: causes of stress 
We wanted to see the main causes of stress at Volvo Trucks AB Umeå. So we found several 
reasons why employees feel stressed at the workplace, the reasons mainly were inability to 
manage  time,  work  overload  and noise  as  the  main  stressors  at  the  workplace.  But  it  is 
necessary  to  say  that  noise  is  subjective  only  to  our  study  as  we  did  our  case  study  on  a 
manufacturing  company  and  the  conclusion  that  noise  is  the  main  causes  of  stress  in  the 
workplace might not be valid beyond an industry that is of manufacturing. Work overload is 
also another main stressor because it puts the employee under pressure to perform too many 
tasks under limited time.  
 
As Volvo Trucks AB Umeå is a manufacturing company we were able to conclude that in a 
manufacturing  multinational  company  where  the  physical  environment  can  be  chaotic  and 
noisy, this situation can be the cause of stress. The conclusion that can be reached from the 58 
 
theories in relation to our purpose is that the workplace causes of stress are work overload, 
poor  working  conditions  such  as  overcrowded  working  conditions  and  noise.  From  our 
analysis we are also able to conclude that the environmental factors of stress are not salient in 
a  multinational  manufacturing  company  such  as  Volvo  Trucks  AB  Umeå  where  the 
employees  were  focused  on  the  internal  factors  of  stress  rather  than  the  external.    Stress 
factors are not always stable, consistent and similar to a group of occupations. It varies from 
environment to environment, work to work or situation to situation, 
Conclusion 2: Steps towards stress management 
We  also  add  to  our  findings  regarding  the  employees  stress  coping  styles  and  steps  that 
employees at Volvo Trucks AB Umeå seek out to their close relatives, friends and families for 
support and consultation. Sharing of feelings and emotions contribute to relieve stress  also 
provide  them  to  enjoy  from  their  professional  life  and  personal  life.  Being  able  to  share 
feelings with peers and families gives different perspective about the proper way to tackle a 
problem  besides  having  a  sense  of  companionship  leads  to  comfortable  workplace. 
Employees  mostly  use  these  tactics  during  the  periods  in  which  they  feel  stress.  However 
stress reduction depends on how the employees manage their time effectively and also how 
the  managers  make  the  workplace  stress  free.  We  also  conclude  that  stress  is  highly  self-
controllable and those employees have the ability to control their feelings and manage their 
stress and for the rest they can refer to the facilities that they are provided by the management. 
Lastly  we  would  like  to  conclude  that  sharing  values  concerning  ambition  level  is  vital  in 
order to experience a reduced stress level. In some instances the employees stress level will be 
affected if they do not know anything about their fellows or close colleagues who works in a 
team. It makes them to be afraid if their ambition level doesn’t match. Therefore it is vital to 
as much as possible to get to know each other’s better and share values and ambition level; 
this will thus affect the stress level and can help in stress reduction.  
 
Conclusion 3: Management of stress  
Lastly  we  add  to  this  conclusion  our  findings  regarding  the  stress  management  of  Volvo 
Trucks AB Umeå. We see that Volvo Trucks AB Umeå has provided a safe workplace for the 
employees.  It  obtained  an  international  award  because  of  having  a  safe  workplace.  Volvo 
trucks AB Umeå has provided medical health center for the treatment of employees stress. 
Employees have freedom to meet the therapist and psychologist as a reason of improving their 
self-esteem and assertiveness. Volvo trucks AB Umeå has also made free time activities and 
various sport facilities for the employees. We think that it is an effective way to support its 
employees in reducing their stress level.  
 
We have also found out from the employees opinion who contributed to this study that stress 
at  the  workplace  is  manageable  but  a  combination  of  both  family  and  work  stressors  are 
highly negative. What we found from the employees open-ended questions is that sometimes 
stress at a certain extent affects positively to their work performance. It makes the employees 
to focus on time management and provide them to render their on job and off job activities 
adequately.  
 
Employee’s stress can be managed by proper time management, seeking help from Human 
Resource Management. Emotion focused strategies like leisure activities, companionship and 
exercise can also be used to relieve stress. Management of a company  also plays an important role in evaluating and managing the stress level of employees at the workplace and should  
use different methods to minimize the stress such as conducting training courses to assist the 
employee’s skills, providing better working environment and making sure that the employees 
get proper guidance and consultation when it’s needed.  
 
To  some  up;  while  many  employees  struggle  with  stress,  in  worst  cases  it  leads  to 
uncertainties and severe impairments  on their health and performance.  The main situations 
that  generate  stress  are  likely  uncontrollable,  unpredictable,  and  are  not  known.  But 
alternatively there are several resources available like personal awareness in coping skills. For 
example:  time  management,  assertiveness,  ways  to  higher  up  self-confidence  and  so  on. 
Management can also utilize some resources for reducing the stress level of the employees by 
investing in training programs, work infrastructure, improving the efficiencies in management 
and employment practices and also trying some other ways which is profitable in organizing 
the work. 


Conclusion
In this thesis, an in-depth literature study was made as a support to two separate experimental studies. In the first experimental study, different stress measures derived using ECG were investigated and compared among each other.  The purpose was primarily to see if they show significant changes in stress load during a period of recovery, but also to compare them during this period. Forthe second experimental study, a real-time solution was created to investigate two respiratory components in an ECG-signal. This was done by monitoringthe two respiratory components during a controlled breathing exercise.  Thepurpose was to find out which one of these followed the breathing pattern themost, and therefore, would be most suitable for biofeedback. This was done bythe use of time and frequency domain algorithms.The first experimental study showed promising results for the stress mea-sures, where all stress measures detected a significant decrease ofstress loadduring a period of recovery. Which could indicate that all measurescould po-tentially work as parameter to identify stress. The most significant decrease wasgiven by the LF/HF ratio, which was taken from the frequency domain. How-ever, more steady decrease was shown for the RMSSD and heart rate signal.Thesuggested stress measure for further use was the LF/HF ratio. However, a stressmeasure based on the frequency domain of the signal would be problematic forreal-time measurements, thus, RMSSD and heart rate were implemented to thesecond study.By analyzing the plain signals, both respiratory components in the secondexperimental study followed the breathing pattern, more of less. They both hada specific feature which made it problematic for the algorithms to identify thecorrect amount of breaths. Compared to the heart rate signal, the R-peak am-plitude signal had a more uneven appearance with more noise. A feature whichwas present for the heart rate signal, that could not be seen for the R-peakamplitude signal, was baseline variations of the signal. During ECG measure-ments, the R-peak amplitude varies around zero, since the values represents thedifferences between R-peak amplitudes. The heart rate on the other hand maydiffer during the measurements due to varying heart rate. Both these signalappearances were also seen in the frequency analysis.Combining the algorithms and the frequency analysis, the heart rate signalfollowed the breathing pattern more accurately in comparison to the R-peakamplitude.With today’s stressful environment with increasing numbers ofstress-relateddiseases, it would be very interesting to implement these stress measures in awearable device. This would allow people to keep track of their stress, to beable to gain more knowledge and learn how to control it. Which hopefully couldbring a more healthy life to many stressful individuals.
