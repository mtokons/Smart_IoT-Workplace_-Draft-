\chapter{Discussion}
\textit{This section I discuss about my system performance and usability in terms of research question . Also mention the limitations of my proposed solutions.}
\vspace{5mm}

With  this  work,  we discussed  a  stress management  system  that  combines  chatbot  and advanced  feature  integration IoT devices to collect and understand various features about the user. In this way, the  bot  can  understand  how  the  user  feels  with  their stress condition and trigger the right action that can better serve user preferences. Moreover, the bot activity dashboard triggers and notifies a  healthcare  expert to  intervene  whenever relevant.
There  was  no study, to  our  knowledge,  that  adapted  smart \acs{IoT} to investigate chatbot, sensors, physical activity intended for work place stress anxiety management specially who have Stress Overflow. We believe there is  a  room  for  improvement  with  respect to  bot  effectiveness  and  IoT integration.  

This study has mapped our research questions. While conducting the literature review, it has been noticed that there is not a lot of work done in stress management at workplace factor of the user in the category of the applications which is being considered as one of the most delicate areas in our real life, we are pointing towards work life, health,and fitness categories. Normally, the employee who have stress will not likely go to the doctor. Stress is one of the most important requirements of this domain and we have understood that it is equally important for any software application which is being developed to target this domain. The results which we have found are noteworthy and can help the people who will be developing stress management apps in the future targeting any kind of disability. 

\section{Research Questions vs Schneckenhaus}

Schneckenhaus system has been developed as a part of this thesis which was focused on workplace stress management and empowerment who have Stress Overflow but the dimensions of stress less working which has been implemented in the system to increase performance are more general and can be applied to any stress management system. We have focused on research questions to evaluate whether the implementation of the dimensions of stress management with smart IoT and app helped stress disability to manage stress or not? The research questions which have been decided in the Introduction chapter of this thesis are as follows: 

Q1: How to manage Stress Overflow in workplace who have stress disability using smart \ac{IoT} and modern technologies?

Q2: How this system will empower people with stress disability to reduce the level of stressful situation, prevent its appearance and cope with stress feelings.

In regards to the whole "Schneckenhaus" application system design and developed in our thesis. we think here are the reasons how it will be helpful to achieve our research goal to manage workplace accessibility for people with stress disability:

- when some one feel stressed at work place, they can use different IoT devices which will be working as conversational ice breakers. When we're thinking about disorders, there's a lot of anxiety about how to handle the subject and not raise uncomfortable questions or don't want to talk about private details. This anxiety can lead to a certain form of silence around the subject that is not very productive for anybody involved. But with the Schneckenhaus, future management and colleagues can demonstrate interesting architecture and different functions there, and this can start a good conversation on how to achieve good accessibility and teamwork in the workplace.

Wear a Schneckenhaus sign badge, light up the eye catcher and the wireless display will be the next step on having good communications between user and their colleagues. Because stress sometimes makes it hard for then to speak and process conversations, it is really a huge help for them to have a non-verbal communication device. It will also be helpful to their colleagues, because they will know that user can't talk at the moment and wont be confused by users behaviour.

The interactive chat bot with activity analytic dashboard will be a huge help to user, to get a better overview when they should take a break and what kind of activity they should perform which is useful for user in a given situation. That kind of long term assistance is really helpful for user, to make the whole "taking a break for anxiety reasons" situation less awkward for them and to make them more confident specially who have stress  disability. Which of course means when they feel more confident,they can also communicate better to their colleagues and empower themselves through out the system. They can handel any situation without any anxiety and there will be less confusion about what is going on with them.

In summery, we believe the "Schneckenhaus" system will be a huge contribution to making future workplaces more accessible for stress disability people because it reduces confusion with their colleagues, therefore reduces stress that miscommunications might cause. And it helps them to analyse their own anxiety/break needs, therefore its motivate them to become more confident in knowing their limits and taking breaks for relaxation at the right time which leads to improving the quality of work and health.

After using our system at work place user will also get positive mental attitude which will empower them to be success in their future career. Here are some positive mental attitude which you will get from activity dashboard:
    \begin{itemize}
        \item Learn to be grateful for what you have – (You can lower stress by putting things in perspective – It could be worse!)
        \item Enjoy your achievements – (You can lower stress by appreciating what you have accomplished so far)
        \item Plan good things for your future (Trips, goals, etc.)
        \item Acceptance – (Learn to accept the things you cannot change)
        \item Cognitive Re-framing – (Is it really that bad? Are you possibly blowing things out of proportion based on emotion?)
        \item Anger management – (Anger and stress can go hand in hand, Learn to cope with anger and watch stress decrease)
    \end{itemize}
    
In the next section we will discuss about limitation of our study.


\section{limitation of the study}
The focus of this research paper is only on employees who work with stress disability in their workplace . We had an aim to understand the causes and management of stress of employees concerning a specific disability group therefore it will not consider the whole group of working people.So the scope of our study is within that disability group and can not be applied to all people.

We used different kind of open source technology platform to make our system more effective and functional but commercial production it will charge lot for making this kind of application. Further development of this system need more research and fund to integrate all functionality in single platform. 


