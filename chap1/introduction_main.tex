\chapter{Introduction}
\textit{In the introductory chapter I will introduce the  overall  theme  of  my research area  and research  topic. First,I start with introducing my thesis background;  afterwards  the motivation of this research. Then the research problem formulation and the research goal are presented. Further, the purpose of my research is also addressed in this part. In this section I also give an introduction to my target People with stress disability and  give  a  historical  overview  of  their  difficulties, this section is concluded after a short thesis delimitation and disposition}
\vspace{5mm}

\section{General Background}
The \acf{IoT} is the modern world's framework in which people and devices interact and communicate with each other.  The human aspect of the Internet of Things contributes to its position in the healthcare sector and can improve health behaviour. To date, numerous technical approaches have been developed with the goal of improving healthcare and carrying out preventive measures \citep{Trmcic2017InternetWell-being}, including stress management in various areas.

Arousal and anxiety can be caused by stressful situations.  The body is under stress when a person is excited. Stress stimulates the sympathetic nervous system, and its activation triggers various reactions in the human body, such as sweat production, increased heart rate, and muscle tension. While stress plays a positive role in success, it can have negative effects on too much stress or repetitive stress. Stress has been recognised as one of the major healthcare problems and has a high impact on the health of individuals. Long-term persistence of stress symptoms in any individual can be used as an indicator of other health conditions and disorders.

Stress management is a challenge for all people, as it can cause anxiety, headaches, and stress, to name only a few issues. We all know that when stressed out, we are less able to manage difficult situations well. Think about the last time you had a bad day at work, was that good when you got home? It's no different for many of the people we're helping.

However, people with mental disability often have a much more difficult time dealing with stress than others. There are a couple of reasons. First, People with stress disability often have more stressful lives than normal individuals. Second, also they don't have as many good things going on in their lives to support them handle stressful things. Lastly, they have not learned how to manage stress (or have had the opportunity to learn). All of these things go together to say that in their lives, people with a mental disability don't have as much "wellness".

I will introduce some approaches in this thesis.That can be use to enhance stress management among mentally challenged people. The aim of practising stress management is to empower anyone to be more relaxed and to build skills so that they can better interact with stressful events when that's happen.

One of the keys to manage stress is to educate people on how to relax. Some people simply don't know how to relax, or they may have only one calming technique that may not work in all circumstances. Stress management is a key element of mental well-being.

\section{Motivation}
Work stress is of major concern to corporate administrators, staff and other stakeholders. Work stress has been defined as the interaction between the individual and the environment, according to Lazarus and Folkman's cognitive theory of stress and coping\citep{LazarusR.S.Folkman1984StressCoping.}. This hypothesis proposed that if environmental demands surpass the resources available, the outcome was either stress or coping, depending on the stressors assessment of the person.

Industrialized countries have undergone significant shifts in the functioning of labour markets in recent decades. Increasing competition on the commodity market, higher aggregate demand volatility and rapid technological development have all led to higher worker productivity pressures. 

At the aggregate level, the latter has been followed by reforming labour market regulation and working arrangements — i.e. decreasing employment protection regulations and implementing non-standard work arrangements — and, at the firm level, growing competition for worker efficiency — i.e. with more demanding job tasks and less supervision of staff\citep{Cottini2013MentalEurope}.Nevertheless, the evidence available supports the concept that working conditions, as well as subjective workplace well-being, have gradually deteriorated in most European countries\citep{Llena-Nozal2009TheCountries}. These changes, among other factors, are likely to affect the health conditions of workers and their overall well-being.

In addition, while the impact of working conditions on health has historically been measured in terms of physical and environmental problems, the change to service jobs and the computerization of work tasks have greatly increased the importance of psychological and mental problems \citep{Cappelli1997ChangeWork}\citep{Robone2008ContractualSurvey}.

Mentally changed people are not perceived the surrounding work environment like normal human. Generally, We use their previous experiences to detect and classify the Stress Overflow. Therefore, by using a previous experiences all the times, it is quite difficult for the people with mental disability to detect and classify Stress Overflow correctly.

\section{Problems Statement }

Stress is the key factor for mental health disorder. The impact of stress problems at the workplace, however, has serious consequences not only for individual well-being but also for company's productivity. Stress is likely to have significant externalities also on other workers, as well as the person with the illness. Employee performance, rates of illness, absenteeism, accidents and staff turnover are all strongly associated to employees mental health status. Workers with better psychological well-being are generally more productive,less likely to suffer from illnesses limiting their working capacity and are less subject to sickness leave. 
In this respect, the burden of mental health disorders on health and productivity has long been underestimated. The economic cost of mental health problems, including treatment and the indirect cost of lost productivity and days of work, is estimated at more than 2 percent of GDP in the United Kingdom \citep{Cooper2005Happiness:Science} and at approximately 1.7 percent of GDP in Canada \citep{stephens}.

Stress are not always bad. In the short term, they can help you overcome a challenge or dangerous situation. Examples of everyday stress include worrying about finding a job, feeling nervous before a big test, or being embarrassed in certain social situations. If we did not experience some stress we might not be motivated to do things that we need to do (for instance, studying for that big test!).

However, if stress begin interfering with your daily life, it may indicate a more serious issue. If you are avoiding situations due to irrational fears, constantly worrying, or experiencing severe anxiety about a traumatic event weeks after it happened, it may be time to seek help.

\section{Research Goal}
The goal of this research is to show the impact of the stress on the work process and the importance of the Stress Management program in the workplace for the People with stress disability. There are a lot of different types of stress; however, there is a concentration on workplace Stress Overflow in this thesis.

The developmental needs of such a research are formulated in three main reasons:
\begin{itemize}
    \item Prevention of stress situations for mentally challenged people.
    \item Reducing the Stress Overflow impact on the work process.
    \item Deliverance from stress anxiety at workplace.
\end{itemize}
The central objectives of this thesis research are: to managing the most frequent reasons of stress anxiety at workplace and empowering People with stress disability, since the enormous amount of all the reasons is impossible to describe in one thesis; to find out the most effective ways to prevent appearing stress anxiety and to fight with it, because no research is so needed as the research with practical pieces of advices; and to show the role of communication when talking about Stress Overflow at workplace.

Topic is worth to be researched mostly because the problem of Stress Overflow, meaning negative effect on work process,occurs more and more often. There are several reasons for presence of such a fact, which I illustrate in my thesis. mentally challenged people are interested in learning how to manage and avoid stress. That is why all the people, who care about their future in workplace, will benefit from stress management research. 

\subsection{Research Question}
During my thesis research there was the continuous retrieval of the answers to the main questions, I have stated the most important ones.

The leading question is about: How to manage Stress Overflow in workplace who have stress disability using smart \ac{IoT} and modern technologies?

The second, but not the less important question, investigated in my thesis, includes sub questions: How this system will empower people with stress disability to reduce the level of stressful situation, prevent its appearance and cope with stress feelings.

Two  important  concepts  on  this  theme,  which  will  be described in my work, are  models  of stress management: Correction Model and Corrective action Model. Those models help to choose the correct way to develop an effective Stress Management system.

\section{Structure of the Thesis}
The thesis report is organized into eight chapters. Chapter 2 illustrates the related works closed to this thesis, chapter 3 covers the concept and solution approach of research questions, chapter 4 describes system overview, chapter 5 include results of different experiments, chapter 6 discuss about research goal vs result, chapter 7 describe the outlook of possible future work and the final chapter contains the conclusion. In appendix the usages of the developed hardware and software tools are given.
