\chapter{Experiment and Result}

\textbf{In an ideal world, you should have two kind of evaluations.  The first is against some ground truth (perhaps a random model?).  The second kind of evaluation is against other people's work (accuracy, speed, etc.).  Any dimension which is of interest, should be evaluated.  Evaluation should be statistically sound. }


\section{Participants}
\section{Duration}
\section{Qualitative Feedback}
\subsection{Jan's Feedback}
In regards to the whole "Schneckenhaus"-ensemble, I think here are the reasons how it will be helpful to achieve more workplace accessibilty for me:

- I think the different devices will be working as conversational "ice breakers". When we talk about Illneses, there is a lot of anxiety on how to approach the subject and to not ask insensitive questions or not wanting to talk about private medical history. This anxiety can lead to a kind of silence around the subject that isn't very productive for anyone involved. But with the "Schneckenhaus"-ensemble I can show future bosses and colleagues there cool design and different functions and this can start a good conversation on how to achieve good workplace accessibilty and teamwork.

- the light-up display will then be a next step on having good communications between me and my colleagues. Because stress sometimes makes it hard for me to speak and process conversations, it is really a huge help for me to have a non-verbal communication device. It will also be helpful to my colleagues, because they will know that I can't talk at the moment and wont be confused by my behaviour.

- and the chat bot with experimental sensors will be a huge help to me, to get a better overview when I should take a break and what kind of break is usefull for me in a given situation. That kind of longterm assistance is really helpful for me, to make the whole "taking a break for anxiety reasons" situation less awkward for me and to become more confident in my disability. Which of course means when I'm more confident I can also communicate better to my colleagues what my deal is and there will be less confusion about what is going on with me.

In short: I think the "Schneckenhaus"-ensemble will be a huge help in making future workplaces more accessible for me because it reduces confusion for my colleagues, therefore reduces stress that miscommunications might cause. And it helps me to catalogue my anxiety/break needs, therefore it helps me to become more confident in knowing my limits and taking breaks at the right time which leads to improving the quality of my work and my health.
