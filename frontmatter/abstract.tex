\begin{abstract}
It’s no surprise that most of us have workplace stress in our lives. But some people are much more struggling at managing with workplace stress than others. Everyone of us has the same physiological response to stress: a release of natural chemicals in the body instantly amplifies our strength and senses to help us act. For our survival this fast reflex has been encoded. But what happens when the same collection of responses the body invokes to the types of everyday stressors, that are features of modern life? There is a need for systems to dynamically interact with workplace stressed people to gather information,  monitor stress condition and provide support, especially at the workplace who have stress overflow or at-home settings. 
This research shows the development of a \acs{IoT} system for workplace stress management. The \acs{IoT} system is developed in an open architecture and Several smart \acs{IoT} devices have been delivered by Smart sensors, stress management application, smartphone, activity dashboard and other digital services. While such solutions offer personalised data and suggestions,  the real disruptive step comes from the interaction of the new digital ecosystem,  represented by sensors, \acs{IoT} devices, chatbot and dashboard. The system is composed of three elements: the one that enables measurement of vital parameters for verifying stress at work time and acknowledged to others, then interact with Chatbot who play a leading role by embodying the function of a virtual friend and bridging the gap between employee and workplace condition and the other activity dashboard for stress management analyses. The stress management program includes a mobile application with relaxation content and IoT platform control. Such a system should minimize the excitement and have an impact on reducing future stress. The IoT system for stress management was evaluated in a real environment, during the work hour at the workplace. The results show that time spent using stress management application with relaxing content and IoT device control can reduce workplace stress who have stress overflow during the working condition. Also, they can learn to manage stress through proven approaches that can enhance their mental health and happiness.
\end{abstract}