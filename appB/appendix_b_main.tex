\chapter{Installation Instructions}
\textbf{\textit{Remote controlled digital display with Screenly OSE and Raspberry Pi}}

\section*{Installing Screenly OSE on the Raspberry Pi with balena} 

Below we’ll walk you through each of the steps that are required for setting up your software and hardware. Pay attention!
\subsection*{Sign up for a free balenaCloud account}

For users with under 10 devices, balenaCloud accounts are free. You can sign up here with a unique username and password or you can sign in with an existing GitHub or Google account.

\subsection*{Create a balenaCloud application}

Once you’re inside your balenaCloud account, click the green “Create application” button near the top left-hand side of the interface. Next, give a name to your application (any name will do, just make a note of it for later!). Also, select the model of your Raspberry Pi device. You can select the “Starter” account package, as this account type is free and will have everything you need for your first few screens. Once you create the application, the main dashboard for the application will load.

\subsection*{Add a device and download the balenaOS disk image from the dashboard}

After creating the application, the next step is to add a device. Click the “+ Add device” button on the top left section of the application interface. You will then see several options. Here’s some guidance on what to select:

    Select your Raspberry Pi model.

    Use the recommended build version of balenaOS.

    Select the “Development” edition of the recommended balenaOS version. The development edition has a variety of testing and troubleshooting tools that are often useful. Read more about the differences between the development and production editions here. If you're confident you can go ahead and deploy the production image instead.

    If you plan to connect your Raspberry Pi to a WiFi connection, enter your WiFi username and password details. It's a good idea to enter your WiFi network details even if you initially only plan on connecting via Ethernet, as it will save time if you wish to switch to WiFi in the future.
    
Once you select the configuration options, click the blue “Download balenaOS (MB)” button. This will download the custom balenaOS disk image to your desktop. This image is automatically built with the information required to connect your device with the application you just created so it's important to use the download from the dashboard rather than anything you've downloaded separately.

\subsection*{Flash your SD card with the balenaOS disk image}

After you have successfully downloaded the balenaOS disk image, you now need to flash this disk image to your SD card. Balena’s very own balenaEtcher is a great free tool for this purpose.

To flash your SD card, connect your SD card to your computer via an internal or external SD reader/writer drive. Open balenaEtcher, select the recently downloaded balenaOS disk image, and select your connected SD card as the destination drive. Next, click “Flash” to write the disk image onto your SD card. This writing process can take several minutes. Be patient!

After you successfully flash balenaOS onto your SD card, safely eject the SD card from your computer and insert the SD card into your Raspberry Pi. Next, connect your Raspberry Pi device to a power outlet. The device should then boot up with the balenaOS operating system. Once the device boots up, you will see the device listed as “Online” on the balenaCloud interface.

If you don’t see the device online, check out balena's extensive troubleshooting guide. Also, check out the forums, as they have a wealth of useful information on any recent issues and the team are available to answer any questions.
Install the balena CLI tools on your workstation

If you already have (or can setup) npm on your machine, using the balena-cli npm package is most likely the easiest way to get the CLI tools up and running quickly. However, there are also standalone binaries and installers for Windows, macOS, and Linux available.

The documentation for the CLI tools is the best place to start, and covers the installation and setup of both the npm package and the standalone binaries.

When you have the CLI installed and working, the first step is to login to balenaCloud by issuing the balena login command:Once you’ve reached this point, and have a working CLI which has been logged in to your account, you're ready to start pushing code to your Raspberry Pi.
\subsection*{Downloading the project from Github}

The next step is to download the code for this project from GitHub. Go to: https://github.com/Screenly/screenly-ose and download the project.
The blue button will download a .zip file of the project which you'll need to unzip, but if you're already familiar with Git you can use git clone in the normal way.
\subsection*{Pushing the project code to your Raspberry Pi}

As you have the CLI setup and the latest code downloaded, you can now execute a single command to push that code to balenaCloud which in turn builds the Docker image and handles the process of setting it up and running it on your device.

From within the unzipped project directory, execute balena push <appName>, where appName is the application name you set back at the beginning of the guide. For example: balena push screenly.

Note: this push command can take around 15 minutes. This time-intensive step builds a disk image for you on the balenaCloud servers. Balena then pushes the built disk image to the balenaCloud Docker registry. Your Raspberry Pi device then pulls the disk image from your balenaCloud account. There’s a lot going on! If this step is successful, you will see an image of Charlie the unicorn appear in your terminal.

If you're using Windows the use of git clone and balena push can cause issues due to line ending changes; we recommend using a combination of either the zip download of the project and balena push or if you'd like to use git clone then also use the git deployment method git push instead of balena push.

At this point, make sure that your Raspberry Pi is connected to your display screen via HDMI. Wait a few more minutes after you see the unicorn, and your display screen should then show a Screenly logo and a URL address. You can check the status of the software download from within the balenaCloud dashboard. At the displayed URL address, you will be able to access the Screenly OSE interface. 

\section*{2. Configuration options}

Once you have Screenly OSE deployed on your device, you must adjust the GPU memory of your Raspberry Pi in order to optimize content playback. To do this, click the “Fleet Configuration” tab on the left-side menu on the balenaCloud interface. On the Fleet Configuration menu, scroll down to find the  configuration item. Then, click “activate” to enable the item. Next, click the pencil icon on the right-hand side and change the value to 192. If you need more information there's a guide to setting configuration variables here.

\subsection*{Enable public device URL for remote access}

Another configuration option that is worth implementing is enabling a public device URL. This configuration option will allow you to access the Screenly OSE interface for the device from anywhere with an internet connection. This is useful for situations when a user needs to manage digital signage content when they are away from the digital sign’s local network.

To set up this configuration, click on the hyperlinked device name in the balenaCloud dashboard. Next, click on the toggle icon under “Public Device URL” so that the toggle icon turns green. You can then click the link icon to navigate to the public URL for your device. You may want to bookmark this link for easy access in the future. 

\section*{3. Using the app}

You can access Screenly OSE via the URL address displayed on your screen or via the public URL provided in the balenaCloud interface. Once you are in the Screenly OSE app, click the blue “+ Add Asset” app on the top right. Next, click “Upload” in the popup screen and select the image and video content you wish to upload. Note that you can also add URL links and display live web pages on your display screen.